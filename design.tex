\documentclass[]{article}
\usepackage[T1]{fontenc}
\usepackage{siunitx}
\usepackage{lmodern}
\usepackage{booktabs}
\usepackage{amssymb,amsmath}
\usepackage{ifxetex,ifluatex}
\usepackage{fixltx2e} % provides \textsubscript
% use upquote if available, for straight quotes in verbatim environments
\IfFileExists{upquote.sty}{\usepackage{upquote}}{}
\ifnum 0\ifxetex 1\fi\ifluatex 1\fi=0 % if pdftex
  \usepackage[utf8]{inputenc}
\else % if luatex or xelatex
  \ifxetex
    \usepackage{mathspec}
    \usepackage{xltxtra,xunicode}
  \else
    \usepackage{fontspec}
  \fi
  \defaultfontfeatures{Mapping=tex-text,Scale=MatchLowercase}
  \newcommand{\euro}{€}
\fi
% use microtype if available
\IfFileExists{microtype.sty}{\usepackage{microtype}}{}
\usepackage{graphicx}
% Redefine \includegraphics so that, unless explicit options are
% given, the image width will not exceed the width of the page.
% Images get their normal width if they fit onto the page, but
% are scaled down if they would overflow the margins.
\usepackage{import}
\graphicspath{{graphics/}}
\makeatletter
\def\ScaleIfNeeded{%
  \ifdim\Gin@nat@width>\linewidth
    \linewidth
  \else
    \Gin@nat@width
  \fi
}
\makeatother
\let\Oldincludegraphics\includegraphics
{%
 \catcode`\@=11\relax%
 \gdef\includegraphics{\@ifnextchar[{\Oldincludegraphics}{\Oldincludegraphics[width=\ScaleIfNeeded]}}%
}%
\ifxetex
  \usepackage[setpagesize=false, % page size defined by xetex
              unicode=false, % unicode breaks when used with xetex
              xetex]{hyperref}
\else
  \usepackage[unicode=true]{hyperref}
\fi
\hypersetup{breaklinks=true,
            bookmarks=true,
            pdfauthor={},
            pdftitle={},
            colorlinks=true,
            citecolor=blue,
            urlcolor=blue,
            linkcolor=magenta,
            pdfborder={0 0 0}}
\urlstyle{same}  % don't use monospace font for urls
\setlength{\parindent}{0pt}
\setlength{\parskip}{6pt plus 2pt minus 1pt}
\setlength{\emergencystretch}{3em}  % prevent overfull lines
\setcounter{secnumdepth}{0}

\author{David Asabina}
\date{}

\begin{document}

\section{Position Detection}\label{position-detection}

This document discusses several methods which provide the necessary
means to measure the orientation of the blinds and therefore offer a
solution for a feedback loop.

\subsection{Electromagnetism}\label{electromagnetism}

Magnetic flux density intensity $\mathbf{B}$ and magnetic flux intensity
$\mathbf{H}$ are two measures that give an indication of magnetic forces
at a given point in space.

$\mathbf{B} = \mu\mathbf{H}$

These measures are both vectors, as such both the magnitude and
direction of the field induced by the reference magnets are available to be 
utilized in determining the blind
orientation.

Through the placement of reference magnets, the position of the blinds or 
variation thereof may be determined. Such an arrangement will inadvertently 
require the placement of one or more sensors located sufficiently within the 
vicinity of the reference magnets to be capable of observing the field 
properties caused these magnet's field properties.

\subsubsection{Magnetic Field Strength}\label{magnetic-field-strength}

Hall effect sensors generally measure the magnitude of the magnetic
field at a given point in space. As such the variation of
electromagnetic field strenght $\delta\mathbf{H}$ may be utilized to
determine the proximity of a magnet to the sensor.

This section will further discuss the different strategies to be
utilized in determining the blind orientation.

\paragraph{Single Sensor Arrangement}\label{single-sensor-arrangement}

Aligning a single hall effect sensor in the module housing and a magnet
on the blind aligned as such to set the magnetic field

\subsubsection{Magnetic Field Direction}\label{magnetic-field-direction}
Upon measuring magnetic field direction the directional component of the 
magnetic field vector $\delta\mathbf{\Theta}$ is analysed. Depending on the 
positioning of the magnet and sensor with respect to each other, to demanded
result may be produced.

The HMC15XX series sensors allow the detection of the direction of the 
fluxfield within a plane, at a given point in space. The entire approach 
discussed within this section will be based on a design utilized the 
aforementioned sensor.

\paragraph{Axle Sensing Arrangement}
Mounting the magnet near the axis of the blind would allow us to measure the
orientation of the blind directly provided that the sensor is positioned near
the axle of the blind as well (see figure \ref{fig:centersense}).
\begin{figure}[h!]
  \centering
  %\def\svgwidth{250pt}
  \input{graphics/centersense.pdf_tex}
  \caption{Axle sensing arrangement}
  \label{fig:centersense}
\end{figure}

The HMC15XX sensor is only capable of detection variations within  90$\deg$
intervals. In order to allow detection over wider ranges, two similar sensing
devices are utilzed with a 45$\deg$ phase shift between the outputs of the 
sensor outputs.

%\begin{table}[h!]
%  %\begin{center}
%    \caption{Phase shift between sensor outputs}
%    \label{hmc15xxexpectedvalues}
%    \begin{tabular}{lrrrrr}
%      %\toprule
%      orientation & 0 & 45 & 90 & 135 & 180 \\
%      A & 0 & 45 & 90 & 45 & 0 \\
%      %\midrule
%      B & 45 & 90 & 135 & 90 & 45 \\
%      %\bottomrule
%    \end
%  %\end{center}
%\end{table}

\begin{figure}[h!]
  \centering
  %\def\svgwidth{250pt}
  \input{graphics/edgesense.pdf_tex}
  \caption{Edge sensing arrangement}
  \label{fig:edgesense}
\end{figure}

\begin{figure}[h!]
  \def\svgwidth{350pt}
  \input{graphics/edgestrategy.pdf_tex}
  \caption{Edge sensing strategy}
  \label{fig:edgestrategy}
\end{figure}

$\frac{a}{b}$
\begin{figure}[hbt]
	\input{circuits/hmc1512}
	\centerline{\box\graph}
	\caption{Basic measuring circuitry for the HMC1512 sensor}
	\label{symlabelforquick}
\end{figure}

The LM358 operation amplifier is rated for \si{100}{dB} which translates to 
voltage amplification of factor 100000 since $20\cdot\log_{10}{\frac{\si{100}{V}}{\si{1}{mV}}}= $ \si{100}{dB}.

The HMC1512 sensor produces a signal over a range of \si{180}{mV} provided that 
the magnetic field detected is strong enough (approximataly 80 Oerstad). During
some tests with a Neodymium magnet the observed range varied over a range of 
\si{60}{mV} which is considerably less than the range mentioned in the 
datasheet. Depending on the placement of the magnet and the strength of the 
magnet itself, the observed results could be even less than the observed range 
of \si{60}{mV}.

The ADC to be used has a resolution of 12 bits therefore offering 4096 
quantization levels. With the beforementioned resolution and the notion that 
the HMC1512 sensor delivers a signal within the measurable range over a 
rotation of \ang{180}, there are $\frac{4096}{180}$ quantization levels per 
degree in rotation which results to an accuracy of approximately 5\% per angle 
assuming the signal into the ADC is void of noise and distortion.


%% 60mV signal -> 55x AMP -> 3.3 V signal
%% 30mV signal -> 110x AMP -> 3.3 V signal

\subsection{Sampling}
In figure \ref{outputa_first} the results from the first bridge are 
portrayed as the system rotated the blinds at fixed steps for \ang{180}. 
Several effects that impair the measurement results are:
\begin{itemize}
  \item obstructed blinds
  \item insufficient mechanical transfer due to leeway
  \item neightboring magnetic fields
\end{itemize}

Both obstruction as incomplete mechanical transfer may introduce delays in the
motion transfer in the system. As such, the motion once transfered may 
introduce sudden jumps leading to oscillation of the blinds in the recovery
from the overshoot (caused by the sudden dislodging of the obstruction for
example).

The measurement algorithm will need to compensate for the named irregularities
to be expected in operation.

%%The sample sets used in this section are acquired by connecting the ADC to the
%%output of an amplifier configured to amplify the differential signal by a 
%%factor of 25. The voltage reference for both amplifiers is set to \si{5}{V}.
%%
%%With the sample set from output 1 approximately 153 quantization levels are 
%%covered. The lowest measured means of step samples ($\mu_{min}$) is 3098 
%%while the highest acquired sample means ($\mu_{max}$) equals 3251.
%%\begin{figure}[ht!]
%%  \centering
%%  \includegraphics{graphics/plots/outa_firstset.pdf}
%%	\caption{Amplified measurement results from 1st bridge}
%%  \label{outputa_first}
%%\end{figure}
%%%% avg_min = 3098.000000, avg_max = 3251.000000, avg_range=153.000000
%%
%%\begin{figure}
%%  \centering
%%  \includegraphics{graphics/plots/outb_firstset.pdf}
%%	\caption{Amplified measurement results from 2nd bridge}
%%  \label{outputb_first}
%%\end{figure}
%%%% avg_min = 2763.250000, avg_max = 3075.750000, avg_range=312.500000

According to the sensor manufacturer the voltage generated on the bridge output
of the sensor in accordance to the equation $\Delta{V_a} = V_s S\sin(2\theta)$,
$\Delta{V_b} = -V_s S\cos(2\theta)$ where $V_s$ represents the supply voltage, 
$S$ represents a material constant and $\theta$ represents a reference to the 
magnetic field angle in degrees.

%% source: http://www51.honeywell.com/aero/common/documents/myaerospacecatalog-documents/Missiles-Munitions/HMC1501-1512.pdf
The conversion of the sensor output $\Delta{V}$ into the corresponding angular 
position in degrees $\theta$ is accomplished with the following expression:
$$\theta = \frac{1}{2}\cdot\arcsin(\frac{\Delta{V_a}}{V_s S}) = \frac{1}{2}\cdot\arccos(\frac{\Delta{V_b}}{-V_s S})$$

Assuming that $V_\phi$ represents the offset observed at the input of the 
amplification stage (caused by common mode offsets or other anomalities)
and $A$ represents the amplifier gain, the resulting voltage observed on the
ADC should equal $V_{signal}\cdot A + V_\phi$ or $(V_{signal}+V_\phi)\cdot A$.
The first expression is appropriate in case the common mode offset is 
negligible and the output offset is known, whilst the latter expression only
considers the influence of common mode offset. A more complete expression for 
the expected output is $V_{\phi_{out}} + A\cdot(V_{signal}+V_{\phi_{CM}})$.

$$V_A = V_s A_a S\sin(2\theta) + V_{\phi_a}$$
$$V_B = -V_s A_b S\cos(2\theta) + V_{\phi_b}$$

\subsubsection{Data Range}
In order to properly understand the data acquired by the ADC it may be
helpful to be aware of conditions as measured by another instrument. The 
results depicted in table \ref{table:extremes} have been acquired with a 
multimeter. The values were obtained by manually rotating the blinds
and correcting the position to achieve either the lowest or highest observable
read-out.

\begin{table}[h!]
  \begin{center}
    \caption{Extremes as measured with a multimeter}
    \label{table:extremes}
    \begin{tabular}{lrr}
      \toprule
      & A & B \\
      \midrule
      A maximum & 2.54 & 1.71 \\
      B maximum & 1.49 & 2.75 \\
      A minimum & 0.45 & 1.75 \\
      B minimum & 1.60 & 0.99 \\
      \bottomrule
    \end{tabular}
  \end{center}
\end{table}

From table \ref{table:extremes} we witness a range of \SI{2.09}{V} in the 
output of sensor A and a range of \SI{1.76}{V} in the output of sensor B. The
cause of this difference is beyond the scope of this section, but in short it
is likely to be a result of differences in sensor placement and component
imperfections. As the differences between a sensors maximum and minimum value
vary it is important to take the related variables into consideration in 
interpreting the data.

The dataset acquired from sensor A has produced the results depicted in figure 
\ref{fig:a_peak}. The absolute maximum value detected was $3234$. Considering
the fact the acquisition was executed by a 12-bit ADC powered by \SI{3.3}{V}
we could expect the equivalent voltage to be approximately \SI{2.60}{V} ($\frac{3234}{2^{12}}\cdot 3.3$).
\begin{figure}[ht!]
  \centering
  \includegraphics{graphics/plots/a_top_values.pdf}
	\caption{The amplified signal from sensor A with the peaks marked}
  \label{fig:a_peak}
\end{figure}

%%The following results have been acquired with the electronic schema using
%\begin{figure}[ht!]
%  \centering
%  \includegraphics{graphics/plots/closed_1_hall_hist.pdf}
%	\caption{Results for the test setup when blinds are closed}
%  \label{outputa_first}
%\end{figure}

%% 
%% \subsection{Gravitation}\label{gravitation}
%% \paragraph{MEMS Gyroscope}
%% 
%% \subsection{Optics}\label{optics}
%% \paragraph{Bridge}
%% 
%% %% http://demonstrations.wolfram.com/MagneticFieldOfACylindricalBarMagnet/
\end{document}
